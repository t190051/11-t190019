\documentclass{article}
\usepackage[utf8]{inputenc}

\title{課題}
\author{t190019 }
\date{July 2020}

\begin{document}

\maketitle{}
(1)//
$\lim{(h\to0)}f(a+h)//
=f(a+0)=f(a)$//
(2)//
$\lim{(h\to0)}\frac{f(a+2h)-f(a)}{2h}//
=f'(a)$(微分の定義)//
(3)//
$\lim{(h\to0)}\frac{f(a)+f(a+2h)-2f(a+h)}{2h^{2}}//
=\frac{\frac{f(a+2h)+f(a)}{2}+f(a+h)}{h^{2}}//
=\frac{f''(a)}{2}$(微分の定義)//
(1)//
$\frac{1}{n}\sum_{k=1}^{n}{\frac{k^{4}}{n^{4}}}//
=\frac{1}{n^{5}}\sum_{k=1}^{n}{k^{4}//
=\frac{1}{30}(1+\frac{1}{n})(2+\frac{1}{n})(3+\frac{3}{n}+\frac{1}{n^{2}})}$//
また$n\to \infty$とすると$\frac{1}{5}$となる//
(2)//
$\frac{1}{n}(\frac{1}{2}+\sum_{k=1}^{n-1}{\frac{k^{4}}{n^{4}}})//
=\frac{1}{2n}+\frac{1}{n^{5}}\sum_{k=1}^{n}{k^{4}//
=\frac{1}{2n}+\frac{1}{30}(1-\frac{1}{n})(2-\frac{1}{n})(3-\frac{3}{n}-\frac{5}{n^{2}})}$//
また$n\to \infty$とすると$\frac{1}{5}$となる//
(3)\\$\frac{4}{3n^{5}}\sum_{k=1}^{\frac{n}{2}-1}{(24k^{4}+48k^{3}+72k^{2}+12k+5)}//
=\frac{8}{15}(\frac{1}{2}-\frac{1}{n})(1-\frac{1}{n})(\frac{3}{4}-\frac{3}{2n}-\frac{1}{n^{2}})+\frac{4}{3n^{4}}(n-2)(6n-3)+\frac{1}{n^{3}}(n-2)^{2}+\frac{20}{3n^{5}}(\frac{n}{2}-1)$//
また$n\to \infty$とすると$\frac{1}{5}$となる//
(4)//
より高次の近似なので、シンプソンの公式が一番誤差が小さくなる。



\end{document}